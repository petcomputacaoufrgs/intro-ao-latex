\begin{tabular}{| c || c c | c | }
\hline
num & nome & valor (R\$) & quantidade \\
\hline \hline
1 & caneta & 2,00 & 3 \\
2 & resma papel & 22,00 & 1 \\
3 & borracha & 0,50 & 2 \\
\hline
\multicolumn{3}{|c|}{valor total} & 29,00 \\ 
\hline
\end{tabular}
\newpage

No modo matemático do \LaTeX{}, é importante lembrar que
$ a^x+y \ne a^{x+y} $

A equação massa-energia pode ser descrita pela famosa equação
\begin{equation}
    E=mc^2
\end{equation}

A função ReLU é muito utilizada na área de machine learning, e pode ser descrita como
$$
\text{função ReLU} = 
\left\{
    \begin{array}{ll}
        y = 0 & x \leq 0 \\
        y = x & x > 0
    \end{array}
\right.
$$

\begin{teo}[Teorema de Pitágoras]
A soma dos quadrados dos catetos é igual ao quadrado da hipotenusa.
\end{teo}